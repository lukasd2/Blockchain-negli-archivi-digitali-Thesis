\chapter{Introduzione}

Le blockchain sono un paradigma tecnologico che consente l’implementazione di un insieme di concetti fondamentali per lo stato di Internet attuale: la trasparenza, la sicurezza, l’immutabilità e la decentralizzazione. Dall’unione di questi concetti, segue la nascita di un sistema di comunicazione, di gestione delle operazioni e dati in rete senza precedenti. Il complesso di elementi costituenti il paradigma permette di creare e gestire basi di dati distribuite, formate da registri replicati su tutti i nodi partecipanti alla rete. Molti sono i modi in cui è possibile utilizzare le blockchain, il campo delle valute digitali è stato il primo ambito applicativo ma le potenzialità più promettenti e innovative derivano dal fatto che il sistema può essere esteso a un insieme arbitrario di entità rappresentabili mediante la digitalizzazione. 
Una base di partenza per introdurre e descrivere l’utilità delle blockchain può essere indotta dalla loro implementazione della tecnologia dei registri distribuiti\footfullcite{uk-gov-report} (Distributed Ledger Techology, di seguito DLT) che per ragioni di chiarezza dell’esposizione è legittimo considerare situata ad un livello di astrazione superiore al modello di riferimento in questione. 

Strettamente correlate a DLT e centrali per questa tesi sono le nozioni di registro e transazione. Per secoli varie istituzioni tra cui banche e governi hanno utilizzato registri e sistemi contabili per tenere traccia di una moltitudine di operazioni, a partire da registrazioni di natura commerciale e contabile ai passaggi di proprietà, relazioni tra privati e così via. Si parla essenzialmente di transazioni aventi per oggetto beni e servizi. Queste operazioni di scambio tra uno o più soggetti sono regolate principalmente da un corrispettivo monetario il cui valore è garantito da un’autorità centrale.

Un altro modo consiste nel registrare i movimenti a credito e a debito (dare e avere) in un opportuno registro, ad esempio in un libro mastro. Questo tipo di transazioni legate a un registro hanno la caratteristica intrinseca di dover essere mantenute da qualcuno, un intermediario. Indipendentemente che sia un’organizzazione privata o un’autorità centrale, sorge il problema della fiducia necessaria verso chi lo mantiene. Ecco che alla base di queste considerazioni iniziali, di registro e transazione, si introduce la tecnologia delle catene di blocchi (di seguito blockchain) che consente l’implementazione di un registro distribuito di transazioni che non necessita di un'autorità centrale fidata. Come espresso in una delle definizioni: “Le blockchain rappresentano una modalità particolarmente trasparente e decentralizzata per la registrazione di elenchi di transazioni”\footfullcite{eu-parl-paper}.

L’obiettivo di questo lavoro è di presentare in maniera sistematica le caratteristiche principali e lo stato dell’arte del paradigma in modo da poter valutare in maniera consapevole se e quando scegliere questa tecnologia con i rispettivi benefici e complicazioni per quanto riguarda lo sviluppo di applicazioni decentralizzate. In rete è possibile trovare molti articoli e spiegazioni che descrivono le blockchain in maniera più o meno completa. Qui, l’attenzione è posta su alcuni argomenti ritenuti più rilevanti per la programmazione in un ecosistema distribuito. Per inquadrare meglio tutte le caratteristiche del progetto finale si è scelto di presentare il procedimento a partire da concetti alla base della tecnologia fino alla creazione di un'applicazione reale sulla blockchain. 

La struttura di questo lavoro rispecchia una divisione a tre strati \footnote{Una divisione analoga a quella presentata durante la conferenza Devoxx in Belgio, tenuta da Sebastien Arbogast e Said Eloudrhiri, http://chainskills.com/2016/11/21/what-qualifies-as-a-blockchain/}: il primo strato contiene i concetti, l’insieme di caratteristiche alla base della tecnologia. Secondo strato si riferisce alle implementazioni, cioè come questi concetti sono implementati nella rete, ad esempio Bitcoin, Ethereum. Nella terza fase abbiamo le istanze di una particolare implementazione della blockchain, in questo caso, lo sviluppo e pubblicazione dell’applicazione con Ethereum nella blockchain di prova, Rinkeby. Riassumendo, il lavoro è stato svolto seguendo questi passi,

\begin{itemize}
%Poi mettere reference ai capitoli
\item Concetti principali della blockchain presentati nel capitolo \ref{ch:blockchain}: Blockchain: il funzionamento e i concetti principali
\item Implementazioni nel capitolo \ref{ch:ethereum}: Ethereum: Sviluppo di applicazioni decentralizzate
\item Istanza dell’applicazione sviluppata nel capitolo \ref{ch:archivi}: Progetto di Archivi Digitali
\end{itemize}

Nella prima parte di questa tesi viene approfondita l’innovazione portata dalla tecnologia blockchain con le caratteristiche e i vantaggi che essa porta con sé. Viene analizzato l’ecosistema decentralizzato, applicato concretamente per la prima volta e su larga scala nel sistema Bitcoin. A partire da questa base reale si procede astraendo alcuni fattori costitutivi che permettono di contrastare problemi derivanti dalla centralizzazione grazie alla risoluzione del problema dei generali bizantini attraverso un sistema di consenso distribuito.
Di seguito si procede estendendo il concetto di transazione in modo che possa rappresentare ogni sorta di bene rappresentabile, concludendo con alcuni esempi reali di applicazioni.

Nella seconda parte si procede sistematicamente con l’ampliamento del concetto delle transazioni, in quanto atte a rappresentare non solo semplici scambi di valuta ma qualunque oggetto della vita reale. Lo sviluppo di programmi memorizzati sulla blockchain stessa, chiamati contratti (smart contracts) permette che le sue funzioni vengano eseguite automaticamente nelle transazioni. In questa sezione verrà analizzato il processo di sviluppo delle applicazioni decentralizzate con la blockchain Ethereum.

Infine, il progetto sugli archivi digitali. Si tratta di un’applicazione costruita sulla piattaforma Ethereum che permette la catalogazione di opere d’arte minori in maniera distribuita. La documentazione mette insieme il processo di sviluppo illustrato nelle sezioni precedenti per costruire un’applicazione decentralizzata pubblicata sulla piattaforma Ethereum.