\chapter{Ethereum: Sviluppo di applicazioni decentralizzate}

Seguendo la divisione a tre strati: di concetti, implementazioni e istanze definita fin dall'inizio del lavoro, in questo capitolo verrà affrontato l'aspetto dell'implementazione nel caso concreto della piattaforma Ethereum. I concetti spiegati fino a questo punto costituiscono un base decisionale, un punto di partenza per scegliere le implementazioni che meglio si adottano alle proprie necessità. I concetti presentati finora possono essere visti come una sorta di contenitore aperto da cui è possibile “pescare” le proprietà da implementare a seconda delle funzionalità del sistema reale. 

Importante sottolineare che lo stato dell'arte delle blockchain limita e forse rende del tutto inutile il senso di descrivere le implementazioni concrete. Questa affermazione deriva dal contesto attuale, le blockchain nonostante non siano più definibili come una tecnologia fra le più recenti in confronto alle tempistiche nel settore informatico, generalmente non hanno raggiunto una stabilità sufficiente. Usando la terminologia presente nelle analisi e predizioni annuali condotte dalla Gartner\footfullcite{gart} la blockchain si trova attualmente sulla via di uscita dal cosiddetto "hype cycle"\footfullcite{gart2016} \smallskip \footfullcite{gart2018}. Una metodologia usata per rappresentare la maturità, l'adozione e l'applicazione di nuove tecnologie.
La situazione può essere vista come un contesto variabile, di continuo sviluppo con l’emergere di nuove implementazioni, piattaforme e applicazioni. Una ricerca spesso finalizzata al miglioramento e adozione delle nuove tecniche per contrastare le limitazioni attuali del sistema ma pur sempre una situazione, qual è quella presente, in continuo mutamento caratterizzata dall'instabilità e cambiamenti radicali. Tuttavia, questo non pregiudica l’utilità delle cose dette finora in quanto, nel caso peggiore, possono servire anche per comprendere le scelte e gli sviluppi futuri.

La scelta di Ethereum come sistema da presentare, collegato alla successiva costruzione di una applicazione nel capitolo successivo è stata fatta seguendo tra le altre, le motivazioni del paragrafo precedente. Viene perseguita l'idea legata alla correlazione tra il successo e la popolarità di un sistema misurata tenendo conto del numero di partecipanti e sviluppatori coinvolti, gli strumenti di sviluppo e il loro grado di maturità messi a disposizione, la frequenza di aggiornamenti con i relativi rischi connessi. Un insieme di fattori che le nuove implementazioni dovranno cercare di raggiungere per attrarre gli utenti e gli sviluppatori ai loro sistemi. Al momento dello scrivere la piattaforma Ethereum è il sistema più sviluppato sotto questi aspetti.

\section{Ethereum Metropolis}

Ethereum è una piattaforma di sviluppo delle applicazioni decentralizzate. Caratteristiche: 
- Permessi: è una blockchain pubblica, decentralizzata\\
- Hash: funzione di hash keccak256\\
- Struttura dati: Merkle Trees - patricia trees\\
- Sistema di consenso PoW\\
- Smart contracts\\
-- DA FARE\\

\subsection{Caratteristiche in Ethereum Turing completness etc.}

\subsection{Transazioni}

\section{Smart Contracts}

\blindtext

\subsection{Solidity}

\blindtext

\section{Data storage}

\blindtext

\subsection{Ipfs}

\blindtext