\chapter{Progetto di Archivi Digitali}
\label{ch:archivi}

Questo capitolo contiene la documentazione del progetto realizzato utilizzando le tecnologie descritte nei capitoli precedenti. Si tratta di un’applicazione decentralizzata costruita attraverso la blockchain pubblica Ethereum e il protocollo IPFS. Lo scopo principale di questo progetto è la descrizione della metodologia di sviluppo in un ecosistema distribuito. Se per molti programmatori imparare un nuovo linguaggio di programmazione o framework può essere una situazione frequente, lo sviluppo in un paradigma differente richiede una descrizione più approfondita della metodologia e delle pratiche migliori da adottare durante la fase di preparazione e di sviluppo.

\section{Definizione di obiettivi e requisiti dell’applicazione}

L’obiettivo dell'applicazione consiste nello sviluppo un registro contenete opere d’arte definibili come minori che, potranno essere memorizzate in maniera trasparente, permanente e immutabile in un contesto distribuito. 

In dettaglio l’applicazione deve soddisfare i seguenti requisiti e funzionalità:

\begin{itemize}
\item Inserimento di opere d’arte minori nell’archivio con l’utilizzo di sistemi di metadati adatti per descrivere l’oggetto da inserire.
\item Memorizzazione e visualizzazione degli oggetti inseriti.
\item Presenza di una categoria di utenti dotati di permessi di verifica dei dati inseriti nel sistema e la loro approvazione attraverso una votazione.
\item Possibilità di modificare l’oggetto inserito da parte del suo autore mantenendo la storia completa delle modifiche effettuate.
\end{itemize}

\subsection{Regole di Codifica}

\section{Preparazione dell'ambiente di sviluppo}

\section{Sviluppo dell'applicazione}
MVP..
\subsection{Scrittura dei Contratti}

\subsection{Fase di test}

\subsection{Testnet}

\subsection{Programmazione front-end}

\subsection{Ethereum Testnet}

\subsection{Considerazioni finali}


