\chapter{Conclusioni}

Il presente elaborato ha illustrato un possibile percorso pensato come un’introduzione allo studio del paradigma blockchain seguito da una parte pratica relativa allo sviluppo di un progetto basato sulla tecnologia in questione.

Uno degli obiettivi principali di questo lavoro era quello di fornire un'organizzazione logica adatta per l’esposizione di un argomento che, generalmente, potrebbe essere giudicato come complesso. La divisione in tre strati - ciascuno dei quali corrispondente a un capitolo - ha permesso di dividere gli argomenti in sottoinsiemi di elementi costituenti. In definitiva, questi elementi costituiscono un percorso ordinato, introdotto con un approccio dal generale al particolare, adottato per massimizzare la chiarezza dell’esposizione. Una delle sfide nella stesura di questo elaborato è stata la ricerca di equilibrio tra il livello di dettaglio nella descrizione degli argomenti trattati e l'accessibilità di questi ultimi da parte di un lettore non esperto.

La realizzazione di questo elaborato è stata possibile anche grazie al tirocinio formativo effettuato presso l’Istituto di Informatica e Telematica del CNR di Pisa. Grazie a questa esperienza, che ha fornito una preparazione adeguata per poter affrontare l'argomento, è stato possibile proseguire con un’ulteriore ricerca sulla tecnologia finalizzata con la creazione di un progetto concreto.

In merito all'aspetto, prevalentemente teorico, relativo alla descrizione delle blockchain, si è dimostrato quali sono i principali vantaggi e gli svantaggi derivanti dall'utilizzo di tali soluzioni. Da una parte il paradigma offre la possibilità concreta di utilizzare la struttura di “\emph{registri distribuiti}” (DLT) con i possibili vantaggi relativi alla decentralizzazione, alla trasparenza, all'immutabilità e alla permanenza dei dati.
Dall'altra parte, queste caratteristiche, sono strettamente connesse con i costi inerenti alle transazioni e, rispetto ai sistemi tradizionali, con i tempi relativamente lunghi dovuti al funzionamento delle blockchain. Questi costi sono dovuti principalmente all'adozione dei sistemi di consenso e alla propagazione di una copia aggiornata del registro a tutti i partecipanti alla rete.

Durante l’analisi dello stato dell’arte è emerso anche l’aspetto dei problemi relativi alla scalabilità. Nell'insieme, l'analisi degli aspetti potenzialmente innovativi correlati ai costi e alle problematiche delle attuali soluzioni implementative, dovrebbe permettere una valutazione più consapevole, in merito all'adozione della tecnologia, per i diversi casi d'uso. Tutto questo tenendo conto del fatto che si tratta di soluzioni tecnologiche in via di sviluppo che, una volta raggiunta una stabilità sufficiente, potranno essere applicate per la creazione di applicazioni reali potenzialmente innovative. 

Uno degli aspetti innovativi su cui ci si è concentrati in questo lavoro deriva dall'espansione delle blockchain con i cosiddetti contratti intelligenti o \emph{smart contracts}. L'unione di questi elementi dà luogo a un nuovo tipo di applicazioni, le cosiddette applicazioni decentralizzate. In particolare per la parte implementativa delle blockchain si è deciso di usare Ethereum, la piattaforma, al momento, più frequentemente usata per la costruzione di questo tipo di applicazioni.

Per quanto riguarda la creazione del progetto che accompagna questo lavoro, l’obiettivo principale era quello di dimostrare la metodologia, insieme alle \emph{best practices}, con cui è possibile costruire le applicazioni decentralizzate. Il risultato è un'applicazione basata sulla blockchain, per la parte \emph{back-end}, e su IPFS per la parte relativa alla memorizzazione dei dati. 

Grazie a questo progetto è stato possibile applicare concretamente la tecnologia allo stato dell'arte, all'ambito delle scienze umane. Si tratta dunque di un impiego possibile di tale soluzione tecnologica nell'ottica della salvaguardia dei beni culturali. In particolare si è scelto di sviluppare una soluzione informatica che permette di catalogare le opere d'arte definibili come minori. Con questo esempio è stato possibile verificare il funzionamento della blockchain come un’alternativa valida alla risoluzione dei problemi più comuni presenti nei sistemi centralizzati. 

Il progetto ha illustrato la metodologia, un'eventuale applicazione reale di questo tipo, al momento è limitata sopratutto dallo stato dell'arte della tecnologia. Si tratta comunque di una soluzione pubblicata su una blockchain di prova, il passo successivo potrebbe consistere nell'aggiornamento dell'applicazione con l'evolversi della tecnologia e nella raccolta di \emph{feedback} da parte di utenti per poter espandere l'applicazione con nuove funzionalità e per verificare ulteriormente la correttezza delle soluzioni attualmente adottate.

Inoltre potrebbe essere importante considerare l'argomento di questa tesi fondamentalmente per qualsiasi applicazione di tipo informatico, ma soprattutto per la figura dell'informatico umanista, per i numerosi campi applicativi legati al trattamento di contenuti culturali. Lo studio di tale tecnologia potrebbe costituire un vantaggio importante per il mondo del lavoro e per affrontare lo studio delle soluzioni di molti problemi presenti nella società dell’informazione attuale.

In conclusione, il presente elaborato non esaurisce l’argomento ma potrebbe essere una buona base di partenza per lo studio più approfondito dei diversi elementi che ne fanno parte. Le blockchain sono una tecnologia promettente, in via di sviluppo, sempre più da tenere a mente nella costruzione di nuove soluzioni informatiche.