\chapter{Conclusioni}

Il presente elaborato illustra un possibile percorso pensato appositamente come un’introduzione allo studio del paradigma blockchain ed è seguito da una parte pratica relativa allo sviluppo di un progetto basato sulla tecnologia in questione.

Uno degli obiettivi principali di questo lavoro era quello di fornire un'organizzazione logica adatta per una presentazione efficace di un argomento che, generalmente, potrebbe essere giudicato come complesso. La divisione in tre strati, dove ciascun strato forma un capitolo di questa tesi, permette di dividere un dato argomento in un sottoinsieme di elementi costituenti. Nell’insieme questi elementi costituiscono un percorso ordinato, introdotto con un approccio, dal generale al particolare, adottato per massimizzare la chiarezza dell’esposizione. Una delle sfide affrontate durante la scrittura di questa tesi consisteva proprio nella ricerca di un equilibrio tra il livello di dettaglio utilizzato per descrivere gli argomenti presentati e, allo stesso tempo, il mantenimento di un buon grado di comprensibilità (intelligibilità?) anche per un lettore non esperto in materia.

La realizzazione di questo elaborato è stata possibile anche grazie al tirocinio formativo effettuato presso l’Istituto di Informatica e Telematica del CNR di Pisa. Grazie a questa esperienza, che ha fornito una preparazione adeguata per poter affrontare l'argomento, è stato possibile proseguire con un’ulteriore ricerca sulla tecnologia finalizzata con la creazione di un progetto concreto.

In merito all'aspetto prevalentemente teorico relativo alla descrizione delle blockchain e dell'implementazione Ethereum, si è dimostrato quali sono i principali vantaggi e gli svantaggi derivanti dall'utilizzo di tali soluzioni. Innanzitutto, è importante considerare le caratteristiche principali del paradigma blockchain, cioè la decentralizzazione, la trasparenza, l'immutabilità e la permanenza. Queste caratteristiche sono strettamente connesse con dei costi veri e propri legati alle transazioni sulla rete e gli svantaggi inerenti ai problemi della scalabilità.

L'analisi dello stato dell'arte della tecnologia dovrebbe permettere una valutazione più consapevole in merito ai costi sopra esposti 

(implementazioni in via di sviluppo...)

Per quanto riguardo lo sviluppo del progetto...
(considerazioni)

(informatica umanistica per beni culturali) - l'impiego di una tale tecnologia per i beni culturali

(debolezze, potenziali problemi, pubblicazione su mainnet costi e problemi derivanti dal'esportazione di un db?)